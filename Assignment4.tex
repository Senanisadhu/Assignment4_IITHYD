\documentclass[a4paper,12pt]{article}
\usepackage{tkz-euclide}
\usepackage{gensymb}
\usepackage[utf8]{inputenc}
\usepackage{graphicx}

\title{ASSIGNMENT-4}
\author{SENANI SADHU}
\date{\today}
\begin{document}
	\maketitle
	\pagenumbering{roman}
	\section{Question:-}
	\paragraph{In $\Delta$ABC, a=6 ,$\angle$B = 60$^{\circ}$ and b-c=2. sketch the triangle.}
	\section{Solution:-}
	Given, a=6 , $\angle$B = 60$^{\circ}$ and b-c=2.\\
	\subsection{Steps of Construction:-}
	\begin{itemize}
		\item Draw base BC of length a=6
		\item Now, lets draw $\angle$B=60$^{\circ}$\\
		Let the ray be BX
		\item From point B as centre,  cut an arc on ray BX.(opposite of BC).Let the arc intersect  BX at D.\\ 
		(Since b-c=2,(c-b) is negative. So, BD will be below line BC.)
		\item Join CD.
		\item  Now, we will draw  perpendicular  bisector of CD.
		\item Mark point A where perpendicular bisector  intersects CD.
		\item  Join AC.
	\end{itemize}
	\subsubsection{Figure:-}
	\begin{center}
			\begin{tikzpicture}
			\draw(0,0)--(6,0);
			\draw(0,0)--(60:5cm);
			\node at (0,0)[below left]{$B$};
			\node at (6,0)[below left]{$C$};
			
			\node at (2.5,4.33)[above left]{$X$};
			\draw(0,0)--(-120:5cm);
			\draw(-0.5,-1.73) arc (-90:-120:2cm);
			\node at (-0.5,-1.73)[below left]{$D$};
			\draw(-1.1,-1.7)--(6,0);
			\draw(0.7,5)--(2.4,-1);
			\draw(0.4,3.5) arc (130:100:3cm);
			\draw(0.4,4.1) arc (70:20:3cm);
			\draw(2,-1.4) arc (220:250:3cm);
			\draw(3,-1.7) arc (-50:-100:2cm);
			\draw(2.4,-1)--(2.88,-2.8);
			\node at (1,3)[below left]{$A$};
		\end{tikzpicture}
	\end{center}
\end{document}